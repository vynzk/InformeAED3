\documentclass[conference]{IEEEtran}
\usepackage[utf8]{inputenc}
\usepackage{graphicx}
\usepackage{amsmath}
\usepackage{url}
\usepackage{cite}

\title{Título del Trabajo en Español o Inglés}

\author{
    \IEEEauthorblockN{Nombre Apellido}
    \IEEEauthorblockA{
        Universidad de Talca\\
        Facultad de Ingeniería\\
        Email: nombre@utalca.cl}
    \and
    \IEEEauthorblockN{Nombre Apellido}
    \IEEEauthorblockA{
        Universidad de Talca\\
        Facultad de Ingeniería\\
        Email: nombre2@utalca.cl}
}

\begin{document}

\maketitle

\begin{abstract}
Resumen del trabajo en unas 6 a 8 líneas. Debe indicar el objetivo principal, metodología utilizada y resultados clave del análisis exploratorio de datos aplicado a clustering.
\end{abstract}

\begin{IEEEkeywords}
Clustering, Análisis Exploratorio de Datos, Reducción de Dimensionalidad, Modelos No Supervisados.
\end{IEEEkeywords}

\section{Introducción}
Describe brevemente el contexto del problema, la importancia del análisis exploratorio de datos y la motivación para aplicar clustering.

\section{Metodología}
Explica los pasos del proyecto:
\begin{itemize}
    \item Descripción del conjunto de datos
    \item Preprocesamiento (datos faltantes, atípicos)
    \item Análisis de tendencia a formar clusters (Hopkins)
    \item Determinación del número óptimo de clusters (métodos: Silhouette, etc.)
    \item Reducción de dimensionalidad (PCA)
    \item Aplicación de algoritmos de clustering (K-Means, DBSCAN, SOM, etc.)
\end{itemize}

\section{Resultados}
Presenta los resultados obtenidos:
\begin{itemize}
    \item Gráficos de componentes principales
    \item Asignación de clusters
    \item Análisis descriptivo por grupo
\end{itemize}

\section{Discusión}
Reflexiona sobre:
\begin{itemize}
    \item Diferencias entre grupos encontrados
    \item Utilidad para la toma de decisiones
    \item Limitaciones del análisis
\end{itemize}

\section{Conclusiones}
Resume las principales conclusiones del trabajo, cómo puede aplicarse en la práctica y posibles mejoras futuras.

\section*{Agradecimientos}
(opcional)

\begin{thebibliography}{99}
\bibitem{ref1} R. L. Nolan, “Managing the Computer Resource: A Stage Hypothesis,” Communications of the ACM, vol. 16, no. 7, pp. 399–405, 1973.
\bibitem{ref2} Nombre del autor, “Título del artículo/libro,” Nombre de la revista/libro, año.
\end{thebibliography}

\end{document}
